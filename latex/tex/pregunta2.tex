\section{ Pregunta 2 }
	%===============================================================================================
	% ENUNCIADO
	\subsection{ Enunciado }
		\begin{enumerate}[a)]
			\item Verifique numéricamente la \textit{Ley de los Grandes Números}. Para ello, escoja una función de densidad de probabilidad, con alguna media $\mu$ y varianza $\sigma^2$, para las variables aleatorias i.i.d, y grafique la evolución de las realizaciones de $\bar X_n$, es decir $\bar x_n \triangleq \frac{1}{n}\sum_{i=1}^n x_i$, a medida que aumenta el número de variables aleatorias involucradas $n$ ¿A qué valor converge $\bar x_n$? Repita lo anterior considerando alguna función de densidad de probabilidad distinta.


			\item Verifique numéricamente el \textit{Teorema del Límite Central}. Para ello, escoja una función de densidad de probabilidad, con alguna media $\mu$ y varianza $\sigma^2$, para las variables aleatorias i.i.d (omita el caso trivial en que las variables aleatorias son Gaussianas), y estime la función de densidad de probabilidad de la variable aleatoria $Z\triangleq (S_n -n\mu)/\sqrt{n}\sigma$ usando histogramas para distintos valores de $n$. ¿Desde qué valor de $n$ es razonable aproximar $Z \sim \mathcal{ N } (0, 1)$? Repita lo anterior considerando alguna función de densidad de probabilidad distinta.

			\textbf{Observación}: Para construir los histogramas puede usar el comando \textit{hist} en Matlab. Use un número alto de realizaciones de la variable Z y  una ventana de alta resolución. Recuerde normalizar el histograma para que la suma de las áreas sea igual a la unidad. 
		\end{enumerate}

	%===============================================================================================
	% SOLUCIÓN
	\subsection{ Solución }

		\subsubsection {Parte a}
		|

			\begin{figure}[H]
			    \centering
			    \includegraphics[width=0.8\textwidth]{pregunta2_a_beta.pdf}
			    \caption{Awesome Image}
			    \label{fig:2_a_beta}
			\end{figure}
			\begin{figure}[H]
			    \centering
			    \includegraphics[width=0.8\textwidth]{pregunta2_a_gamma.pdf}
			    \caption{Awesome Image}
			    \label{fig:2_a_gamma}
			\end{figure}