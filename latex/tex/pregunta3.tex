\section{ Pregunta 3 }

	\subsection{ Enunciado }
		Considere dos variables aleatorias escalares i.i.d. $X$ e $Y$, cada una con función de densidad exponencial de media $\mu$. Defina 

		$$Z\triangleq \frac{X+Y}{\max \lbrace X,Y \rbrace}.$$

		Determine la función de densidad $f_Z(z)$.

		\vspace{5mm}
		\textbf{Sugerencia:} Determine primero la función de distribución $F_Z(z)$ y derive.
	\subsection{ Solución }

		Primero y para simplificar el análisis posterior se observan los valores factibles de la v.a Z,

		\begin{eqnarray}
			Z = \frac{X+Y}{\max \lbrace X,Y \rbrace}\\
			= \frac{\max \lbrace X,Y \rbrace + \min \lbrace X,Y \rbrace}{\max \lbrace X,Y \rbrace}\\
			=  1 + \frac{\min \lbrace X,Y \rbrace}{\max \lbrace X,Y \rbrace}\\
			\iff Z \in ]0,1]
		\end{eqnarray}

		La última igualdad se justifica dado que X e Y son positivos diferentes de 0 (muy cercanos si se desea pero el 0 no es parte del recorrido de X ni Y).\newline

		Para continuar se hará uso de la sugerencia y se planteará $F_Z(u) = P_r\lbrace Z < u \rbrace$, para plantear adecuadamente las expresiones necesarias se comenzará analizando el caso de igualdad Z = u. 

		\begin{eqnarray}
			Z=u \iff Y = y = \left\{
			  \begin{array}{l l}
			    x(u-1) & \quad \text{Si x $>$ y}\\
			    x/(u-1) & \quad \text{Si y $>$ x}
			  \end{array} \right.
		\end{eqnarray}

		Teniendo estos límites se hace más sencillo entender las regiones en las que se cumple $Z < u$, la siguiente imagen muestra estas regiones, notar que la curva $y=x$ separa claramente ambas regiones. 

		\begin{figure}[H]
		    \centering
		    \includegraphics[width=0.8\textwidth]{pregunta3.pdf}
		    \caption{Las regiones $A_1$ y $A_2$ son las regiones para las cuales Z es menor que u.}
		    \label{fig:3_1}
		\end{figure}

		Ahora que tenemos las regiones es más fácil calcular $F_Z(u)$

		\begin{eqnarray}
			F_Z(u) = P_r\{Z < u\} = P_r\{A_1 \cup A_2\} = P_r\{A_1\} + P_r\{A_2\}
		\end{eqnarray}

		Dadas las simetrías del problema se simplifica aún más.

		\begin{eqnarray}
			F_Z(u) =  P_r\{A_1\} + P_r\{A_2\} = 2 P_r\{A_1\}
		\end{eqnarray}

		Por lo tanto basta con calcular $P_r\{A_1\}$.

		\begin{eqnarray}
			 P_r\{A_1\}=\int_{0}^{\infty} \int_{0}^{x(u-1)}  f_{X,Y}(X=x,Y=y)   dy dx
		\end{eqnarray}

		Dado que son independientes se puede reemplazar la conjunta por las marginales.

		\begin{eqnarray}
			 P_r\{A_1\}= \int_{0}^{\infty}f_{X}(X=x) \int_{0}^{x(u-1)}  \lambda \exp(-\lambda y)   dy dx \\
			=\int_{0}^{\infty} \lambda \exp(-\lambda x)   \left[-\exp(-\lambda y)\right]_{0}^{x(u-1)}   dy dx \\
			=\int_{0}^{\infty} \lambda \exp(-\lambda x)   \left(-\exp(-\lambda x(u-1)\right) + 1) dy dx\\
			=\int_{0}^{\infty} -\lambda \exp(-\lambda xu)   + \lambda\exp(-\lambda x) dy dx\\
			= \left[\frac{1}{u}\exp(-\lambda xu)\right]_{0}^{\infty} -\left[\exp(-\lambda x)\right]_{0}^{\infty} \\
			=1-\frac{1}{u}\\
			\iff F_Z(u) = 2\left(1-\frac{1}{u}\right)
		\end{eqnarray}

		En realidad la expresión anterior fue siempre pensando en que u estaba en el rango de Z, pero esto no es una condición, luego la expresión correcta es.

		\begin{equation}
			F_Z(u) = \left\{
			  \begin{array}{l l}
			    0 & \quad \text{Si u $<$ 1}\\
			    2\left(1-\frac{1}{u}\right) & \quad \text{Si $1<u\leq2$}\\
			    1 & \quad \text{Si $u>2$}
			  \end{array} \right.
		\end{equation}

		Continuando con la sugerencia se calcula la densidad de probabilidad a partir de la acumulada con el operador derivada.

		\begin{equation}
			f_Z(Z=z) = \left\{
			  \begin{array}{l l}
			    0 & \quad \text{Si z $<$ 1}\\
			    \frac{2}{z^2} & \quad \text{Si $1<z\leq2$}\\
			    0 & \quad \text{Si $u>2$}
			  \end{array} \right.
		\end{equation}