\documentclass[11pt,twoside,letterpaper]{article}

\usepackage[usenames,dvipsnames]{xcolor} % Para los colores
\usepackage[utf8]{inputenc}
\usepackage{titlesec} % Se usa para darle color y estilos al formato de las secciones y subsecciones.
%\usepackage{showframe} % Se usa para ver como estan hecho los margenes de manera de debugear.

%Seteando los margenes
\usepackage{geometry}
 \geometry{
 letterpaper,
 left=15mm,
 right=20mm,
 top=20mm,
 bottom=20mm,
 }

%Dandole los colores a las secciones.
\titleformat{\section}{\color{blue}\normalfont\Large\bfseries}{\color{blue}\thesection}{1em}{}
\titleformat*{\subsection}{\normalfont\large\bfseries\color{BlueViolet}}

%Para ver los números romanos:
\renewcommand{\thesection}{\Roman{section}} 
\renewcommand{\thesubsection}{\thesection.\Roman{subsection}}

\begin{document}
	% Página de titulo.
	\title{Tarea 1 IPD 431}
	\author{Pascal Sigel\\pascal.sigel@gmail.com}
	\date{Abril 2015}
	\maketitle
	% Fin página de titulo.
	\newpage
	
	\section{Introducción}
	En el siguiente documento ..
	

	\section{ Pregunta 1 }

	\subsection{ Enunciado }

	\subsection{ Solución }

	\section{ Pregunta 2 }

	\subsection{ Enunciado }

	\subsection{ Solución }

	\section{ Pregunta 3 }

	\subsection{ Enunciado }

	\subsection{ Solución }

\end{document}