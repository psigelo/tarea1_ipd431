\documentclass[11pt,twoside,letterpaper]{article}

\usepackage[usenames,dvipsnames]{xcolor} % Para los colores
\usepackage[utf8]{inputenc}
\usepackage{graphicx}
\usepackage{titlesec} % Se usa para darle color y estilos al formato de las secciones y subsecciones.
%\usepackage{showframe} % Se usa para ver como estan hecho los margenes de manera de debugear.
\usepackage{amsfonts}
\usepackage{amsmath}
\usepackage{amssymb}
\usepackage{amsxtra}
\usepackage{float}
\usepackage{enumerate}
%Seteando los margenes
\usepackage{geometry}
 \geometry{
 letterpaper,
 left=15mm,
 right=20mm,
 top=20mm,
 bottom=20mm,
 }

\graphicspath{ {./img/} } %Es la ruta que se considerara para buscar las imagenes.

\renewcommand{\floatpagefraction}{0.75}
%Dandole los colores a las secciones.
\titleformat{\section}{\color{blue}\normalfont\Large\bfseries}{\color{blue}\thesection}{1em}{}
\titleformat*{\subsection}{\normalfont\large\bfseries\color{blue}}
\titleformat*{\subsubsection}{\normalfont\large\bfseries\color{blue}}

%Para ver los números romanos:
\renewcommand{\thesection}{\Roman{section}} 
\renewcommand{\thesubsection}{\thesection.\Roman{subsection}}
\renewcommand{\thesubsubsection}{\thesubsection.\roman{subsubsection}}


\newcommand{\treq}[0]{\triangleq} % Usado por el profesor en su documento.

\begin{document}
	% Página de titulo.
	\title{Tarea 1 IPD 431}
	\author{Pascal Sigel\\pascal.sigel@gmail.com}
	\date{Mayo 2015}
	\maketitle
	% Fin página de titulo.
	\newpage
	
	\section{Introducción}

	Este documento corresponde a la solución de la tarea 1 del ramo ipd431 realizada el primer semestre 2015, Se expondrán los enunciados y su solución inmediatamente después. Se considerará como trivial cualquier concepto visto en clases lo que significa que no requerirá mayores explicaciones ni el uso de referencias.\newline

	{Todo el contenido necesario para reproducir esta tarea, ya sean los códigos python de la pregunta 2 como el documento latex, se encuentra en el repositorio público de github https://github.com/psigelo/tarea1\_ipd431.}\newline


	Las siguientes secciones serán las preguntas con sus respuestas. No existirán conclusiones por no ser necesarias.\newline
	
	\section{ Pregunta 1 }

	\subsection{ Enunciado }

	\subsection{ Solución }
	\section{ Pregunta 2 }
	%===============================================================================================
	% ENUNCIADO
	\subsection{ Enunciado }
		\begin{enumerate}[a)]
			\item Verifique numéricamente la \textit{Ley de los Grandes Números}. Para ello, escoja una función de densidad de probabilidad, con alguna media $\mu$ y varianza $\sigma^2$, para las variables aleatorias i.i.d, y grafique la evolución de las realizaciones de $\bar X_n$, es decir $\bar x_n \triangleq \frac{1}{n}\sum_{i=1}^n x_i$, a medida que aumenta el número de variables aleatorias involucradas $n$ ¿A qué valor converge $\bar x_n$? Repita lo anterior considerando alguna función de densidad de probabilidad distinta.


			\item Verifique numéricamente el \textit{Teorema del Límite Central}. Para ello, escoja una función de densidad de probabilidad, con alguna media $\mu$ y varianza $\sigma^2$, para las variables aleatorias i.i.d (omita el caso trivial en que las variables aleatorias son Gaussianas), y estime la función de densidad de probabilidad de la variable aleatoria $Z\triangleq (S_n -n\mu)/\sqrt{n}\sigma$ usando histogramas para distintos valores de $n$. ¿Desde qué valor de $n$ es razonable aproximar $Z \sim \mathcal{ N } (0, 1)$? Repita lo anterior considerando alguna función de densidad de probabilidad distinta.

			\textbf{Observación}: Para construir los histogramas puede usar el comando \textit{hist} en Matlab. Use un número alto de realizaciones de la variable Z y  una ventana de alta resolución. Recuerde normalizar el histograma para que la suma de las áreas sea igual a la unidad. 
		\end{enumerate}

	%===============================================================================================
	% SOLUCIÓN
	\subsection{ Solución }

		\subsubsection {Parte a}
		
			Por simplicidad se definió sólo una variable aleatoria la cual tiene varias realizaciones, el argumento para asumir eso es que cada realización en si es una variable aleatoria iid con el resto de los instantes. Se escogieron las distribuciones beta con parámetros $\beta = 1$ y $\alpha = 5$ y gamma con parámetros shape = 5 y scale = 1. Los resultados obtenidos pueden verse en las figuras (\ref{fig:2_a_beta}) y (\ref{fig:2_a_gamma}).

			\begin{figure}[H]
			    \centering
			    \includegraphics[width=0.8\textwidth]{pregunta2_a_beta.pdf}
			    \caption{En la figura de arriba se observa el histograma de varias realizaciones de una v.a con distribución beta de parámetros $\beta = 1$ y $\alpha = 5$, la figura de abajo muestra el promedio de las realizaciones donde el eje x es la cantidad de realizaciones, la diferencia con la esperanza está representado por la zona coloreada.}
			    \label{fig:2_a_beta}
			\end{figure}

			\begin{figure}[H]
			    \centering
			    \includegraphics[width=0.8\textwidth]{pregunta2_a_gamma.pdf}
			    \caption{En la figura de arriba se observa el histograma de varias realizaciones de una v.a con distribución gamma de parámetros shape = 5 y scale = 1, la figura de abajo muestra el promedio de las realizaciones donde el eje x es la cantidad de realizaciones, la diferencia con la esperanza está representado por la zona coloreada.}
			    \label{fig:2_a_gamma}
			\end{figure}

			Como se observa de las imagenes anteriores de en ambos casos el promedio de las realizaciones tiende poco a poco a la esperanza calculada de forma teórica.



		\subsubsection {Parte b}
			Se escogieron las distribuciones beta con parámetros $\beta = 1$ y $\alpha = 5$ y gamma con parámetros shape = 5 y scale = 1. Los resultados obtenidos pueden verse en las figuras (\ref{fig:2_b_beta}) y (\ref{fig:2_b_gamma}).

			\begin{figure}[H]
			    \centering
			    \includegraphics[width=0.8\textwidth]{pregunta2_b_beta.pdf}
			    \caption{Se presentan cuatro imagenes resultantes de cantidades diferentes de variables aleatorias beta usadas para generar la v.a Z. La linea roja punteada corresponde a la densidad de probabilidades de $\mathcal{ N } (0, 1)$}
			    \label{fig:2_b_beta}
			\end{figure}

			\begin{figure}[H]
			    \centering
			    \includegraphics[width=0.8\textwidth]{pregunta2_b_gamma.pdf}
			    \caption{Se presentan cuatro imagenes resultantes de cantidades diferentes de variables aleatorias beta usadas para generar la v.a Z. La linea roja punteada corresponde a la densidad de probabilidades de $\mathcal{ N } (0, 1)$}
			    \label{fig:2_b_gamma}
			\end{figure}

			Primero que todo se observa que a medida que la cantidad de variables aleatorias, n, crece más parecida a parecida a una distribución $\mathcal{ N } (0, 1)$ se vuelve la función de densidad de Z. Además rápidamente con n=10 ya parece gaussiana en ambos casos pero entre más grande n mejor se vuelve la aproximación. \newline

			\textbf{Observación:} {\color{red} Todo el contenido necesario para reproducir esta tarea, ya sean los códigos python de la pregunta 2 como el documento latex, se encuentra en el repositorio público de github} \\ 


			https://github.com/psigelo/tarea1\_ipd431.
	\section{ Pregunta 3 }

	\subsection{ Enunciado }
		Considere dos variables aleatorias escalares i.i.d. $X$ e $Y$, cada una con función de densidad exponencial de media $\mu$. Defina 

		$$Z\triangleq \frac{X+Y}{\max \lbrace X,Y \rbrace}.$$

		Determine la función de densidad $f_Z(z)$.

		\vspace{5mm}
		\textbf{Sugerencia:} Determine primero la función de distribución $F_Z(z)$ y derive.
	\subsection{ Solución }

		Primero y para simplificar el análisis posterior se observan los valores factibles de la v.a Z,

		\begin{eqnarray}
			Z = \frac{X+Y}{\max \lbrace X,Y \rbrace}\\
			= \frac{\max \lbrace X,Y \rbrace + \min \lbrace X,Y \rbrace}{\max \lbrace X,Y \rbrace}\\
			=  1 + \frac{\min \lbrace X,Y \rbrace}{\max \lbrace X,Y \rbrace}\\
			\iff Z \in ]0,1]
		\end{eqnarray}

		La última igualdad se justifica dado que X e Y son positivos diferentes de 0 (muy cercanos si se desea pero el 0 no es parte del recorrido de X ni Y).\newline

		Para continuar se hará uso de la sugerencia y se planteará $F_Z(u) = P_r\lbrace Z < u \rbrace$, para plantear adecuadamente las expresiones necesarias se comenzará analizando el caso de igualdad Z = u. 

		\begin{eqnarray}
			Z=u \iff Y = y = \left\{
			  \begin{array}{l l}
			    x(u-1) & \quad \text{Si x $>$ y}\\
			    x/(u-1) & \quad \text{Si y $>$ x}
			  \end{array} \right.
		\end{eqnarray}

		Teniendo estos límites se hace más sencillo entender las regiones en las que se cumple $Z < u$, la siguiente imagen muestra estas regiones, notar que la curva $y=x$ separa claramente ambas regiones. 

		\begin{figure}[H]
		    \centering
		    \includegraphics[width=0.8\textwidth]{pregunta3.pdf}
		    \caption{Las regiones $A_1$ y $A_2$ son las regiones para las cuales Z es menor que u.}
		    \label{fig:3_1}
		\end{figure}

		Ahora que tenemos las regiones es más fácil calcular $F_Z(u)$

		\begin{eqnarray}
			F_Z(u) = P_r\{Z < u\} = P_r\{A_1 \cup A_2\} = P_r\{A_1\} + P_r\{A_2\}
		\end{eqnarray}

		Dadas las simetrías del problema se simplifica aún más.

		\begin{eqnarray}
			F_Z(u) =  P_r\{A_1\} + P_r\{A_2\} = 2 P_r\{A_1\}
		\end{eqnarray}

		Por lo tanto basta con calcular $P_r\{A_1\}$.

		\begin{eqnarray}
			 P_r\{A_1\}=\int_{0}^{\infty} \int_{0}^{x(u-1)}  f_{X,Y}(X=x,Y=y)   dy dx
		\end{eqnarray}

		Dado que son independientes se puede reemplazar la conjunta por las marginales.

		\begin{eqnarray}
			 P_r\{A_1\}= \int_{0}^{\infty}f_{X}(X=x) \int_{0}^{x(u-1)}  \lambda \exp(-\lambda y)   dy dx \\
			=\int_{0}^{\infty} \lambda \exp(-\lambda x)   \left[-\exp(-\lambda y)\right]_{0}^{x(u-1)}   dy dx \\
			=\int_{0}^{\infty} \lambda \exp(-\lambda x)   \left(-\exp(-\lambda x(u-1)\right) + 1) dy dx\\
			=\int_{0}^{\infty} -\lambda \exp(-\lambda xu)   + \lambda\exp(-\lambda x) dy dx\\
			= \left[\frac{1}{u}\exp(-\lambda xu)\right]_{0}^{\infty} -\left[\exp(-\lambda x)\right]_{0}^{\infty} \\
			=1-\frac{1}{u}\\
			\iff F_Z(u) = 2\left(1-\frac{1}{u}\right)
		\end{eqnarray}

		En realidad la expresión anterior fue siempre pensando en que u estaba en el rango de Z, pero esto no es una condición, luego la expresión correcta es.

		\begin{equation}
			F_Z(u) = \left\{
			  \begin{array}{l l}
			    0 & \quad \text{Si u $<$ 1}\\
			    2\left(1-\frac{1}{u}\right) & \quad \text{Si $1<u\leq2$}\\
			    1 & \quad \text{Si $u>2$}
			  \end{array} \right.
		\end{equation}

		Continuando con la sugerencia se calcula la densidad de probabilidad a partir de la acumulada con el operador derivada.

		\begin{equation}
			f_Z(Z=z) = \left\{
			  \begin{array}{l l}
			    0 & \quad \text{Si z $<$ 1}\\
			    \frac{2}{z^2} & \quad \text{Si $1<z\leq2$}\\
			    0 & \quad \text{Si $u>2$}
			  \end{array} \right.
		\end{equation}

\end{document}